\documentclass{report}


\title{Head First Design Patterns}
\date{}

\begin{document}

\maketitle

\section*{Chapter One - Introduction to Design Patterns}

When designing software, bear in mind some of the following:

\begin{itemize}
  \item Identify what stays the same and what varies
  \begin{itemize}
    \item{Encapsulate whatever stays the same}
  \end{itemize}
  \item Program to an interface
  \begin{itemize}
    \item{Define common methods in superclass and then implement these individually within the class}
  \end{itemize}
  \item Favour composition over inheritance
  \begin{itemize}
    \item Understand when to use a HAS-A relationship instead of an IS-A relationship
    \item Construct objects from different classes over getting them to inherit
  \end{itemize}
\end{itemize}
Also, remember The Strategy Pattern Design Principle which is where a family of algorithms is encapsulated and can be used interchangeably

\section*{Chapter Two - The Observer Pattern}

The observer pattern is analogous to people subscribing to a newsletter - we have a series of objects keeping tabs on any changes for a particular subject

\begin{itemize}
  \item We call the publisher of any new information the \textbf{subject}
  \item We call the subscriber of any new information the \textbf{observer}
  \item A more formal definition of the observer pattern is a \textbf{one-to-many dependency between objects so that when one object changes state, all dependents are also notified}
\end{itemize}

Java has a built-in API for creating observers/observables which has some drawbacks:

\begin{itemize}
  \item Observable has many crucial methods protected so you need to always subclass
  \item Observable is a class so its very hard to add more methods underneath
\end{itemize}

Another golden design principle to remember from this chapter is that \textbf{loosely coupled designs for interacting objects is good}

\section*{Chapter Three - The Decorator Pattern}

The decorator pattern is analogous to finding ways to easily incorporate add-ons to a beverage from a coffee shop

\begin{itemize}
  \item Sometimes, inheritance is not the best choice of design for an object
  \item Fundamentally, classes should be open for extension but closed for modification
  \begin{itemize}
    \item Using decorators allows us to continue adding onto a class appropriately
  \end{itemize}
  \item {More formally, decorators allow us to dynamically attach extra functionality to a class}
\end{itemize}


\end{document}
          
          
          